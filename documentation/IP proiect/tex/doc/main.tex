\documentclass[12pt]{article}
\usepackage[utf8]{inputenc}
\input{structure.tex} % Include the file specifying the document structure and custom commands
\title{Ingineria programării. Proiect: inferența prin enumerare} % Title of the assignment

\author{\textbf{Nicolae Boca, Ștefan Ignătescu, Gabriel Răileanu}\\ \texttt{1409A}}
\date{aprilie 2020}

\begin{document}

\maketitle % Print the title

\section{Introducerea}
Programul de față calculează probabilitățile de apariție ale unor afecțiuni medicale, având în vedere diferite afecțiuni pre-existente. Pe baza probabilităților deja cunoscute, se pot defini noduri observate și noduri evidență.
\par
Proiectul de față își propune respectarea unor standarde în ceea ce privește stilul codului, tratarea excepțiilor, dar și modalitățile în care se realizează testarea unităților.
\par
De asemenea, s-au urmărit toate etapele existente în cadrul evoluției unui produs software (de la proiectare, la implementare, apoi la testare, ciclul încheindu-se cu livrarea artefactului software)
\section{Diagrame UML}
\section{Modul de utilizare al programului}
Pentru a ajunge la scopul final al proiectului (realizarea de operații de căutare pe colecția inițială), s-a realizat 2 tipuri de indecși:
\begin{itemize}
	\item index direct cantitativ
	\item index invers cantitativ
\end{itemize}
Fiecare astfel de index este persistat într-o bază de date non-relațională (în acest caz alegându-se o bază de date orientată document: MongoDB).
\section{Capturi de ecran}
\subsubsection{Concluzii}

\section{Contribuții individuale}
\begin{itemize}
	\item Nicolae Boca:
	\item Ștefan Ignătescu:
	\item Gabriel Răileanu:
\end{itemize}
\newpage
\section{Bibliografie}
\begin{thebibliography}{9}
	\bibitem{hadoopInAction} 
	ceva aici
	
	\bibitem{HDFSBlocks} 
	ceva aici2
\end{thebibliography}
\end{document}

\appendix{Anexa 1}
fsadsdfasf