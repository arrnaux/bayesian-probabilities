\documentclass[12pt]{article}
\usepackage[utf8]{inputenc}
\usepackage[romanian]{babel}
\usepackage{graphicx}
\usepackage{csvsimple}
\usepackage{pdfpages}
\usepackage{float}

\input{structure.tex} % Include the file specifying the document structure and custom commands
\title{Ingineria programării. Proiect: inferența prin enumerare} % Title of the assignment

\author{\textbf{Nicolae Boca, Ștefan Ignătescu, Gabriel Răileanu}\\ \texttt{1409A}}
\date{mai 2020}

\begin{document}

\maketitle

\section{Introducere}
Programul de față calculează probabilitățile de apariție ale unor afecțiuni medicale, pe baza datelor deja existente despre anumite comorbidități. Pe baza probabilităților deja cunoscute, se pot defini noduri observate și noduri evidență.
\par
Proiectul de față își propune respectarea unor standarde în ceea ce privește stilul codului, tratarea excepțiilor, dar și modalitățile în care se realizează testarea unităților.
\par
De asemenea, s-au urmărit toate etapele prezente în dezvoltarea unui produs software (proiectare, implementare, testare și livrarea artefactului software).
\section{Diagrame UML}
Sunt prezentate diagrame UML pentru: cazuri de utilizare, clase, activități și secvențe.
\begin{figure}[H]
	\centering
	\includegraphics{img/useCaseDiagram.png}
	\caption{Diagramă de activitate}
\end{figure}
\begin{figure}[H]
	\centering
	\includegraphics[width=\linewidth]{img/diagrama-activitate.png}
	\caption{Diagramă de activitate}
\end{figure}
\begin{figure}[H]
	\centering
	\includegraphics[width=0.75\linewidth]{img/sequenceDiagram.png}
	\caption{Diagramă de secvențe}
\end{figure}
\section{Modul de utilizare al programului}
Aplicația dispune de o interfață grafică din care utilizatorul poate seta valorile probabilităților pentru afecțiunile disponibile. De asemenea, există posibilitatea importării unui set predefinit de date (cele folosite în cadrul laboratorului de inteligență artificială), dar și opțiunea de a încărca datele dintr-un fișier, care trebuie să respecte un anumit format, ilustrat în tabelul \ref{probabilitiesFileFormatTable}.

\begin{table}[h!]
	\begin{center}
	\caption{Exemplu de format pentru probabilități, în cazul unui nod cu doi părinți}
	\label{probabilitiesFileFormatTable}
\begin{tabular}{c|c|c}
	\hline
	\textbf{first parent}&\textbf{second parent} &\textbf{val(P(x)=true)}
	\\ \hline
	true & true & value \\
	true & false & value \\
	false & true & value \\
	false & false & value \\
	\hline
\end{tabular}
\end{center}
\end{table}
În cazul introducerii valorilor de probabilitate în mod manual, utilizatorul va fi împiedicat să scrie date eronate, iar câmpurile alăturate se vor completa automat pentru a ușura acest proces.
\par
După alegerea valorilor pentru probabilități, utilizatorul poate alege din meniuri de tip radio sau drop-down nodurile de interes (noduri evidențe sau noduri observate).
\section{Testarea aplicației}
Pentru aplicația dezvoltată au fost testate atât funcționarea în parametri	 a interfeței grafice, cât și logica pentru calcularea probabilităților condiționate. 
Aceste cazuri de test sunt prezentate în figura 1.
\begin{figure}[H]
	\centering
	\includegraphics[scale=0.38]{raport}
	\label{fig:testCases}
	\caption{Cazurile de test}
\end{figure}
\section{Capturi de ecran}
	\begin{figure}[H]
		\centering
		\includegraphics [scale=0.5] {img/gui.png}
		\caption{Interfața aplicației}
	\end{figure}
	\begin{figure}[!h]
	\centering
	\includegraphics [scale=0.3] {img/help.png}
	\caption{Captură de ecran din meniul de \textit{help} al aplicației}
\end{figure}
\section{Contribuții individuale}
\begin{itemize}
	\item \textbf{Nicolae Boca}: creare meniu \textit{help}, design cazuri de test și testarea unităților
	\item \textbf{Ștefan Ignătescu}: refactorizare, implementare șablon de proiectare, respectarea stilului de codare
	\item \textbf{Gabriel Răileanu}: documentație, document SRS, coordonare echipă
\end{itemize}
\includepdf[pages=-]{../srs/srs.pdf}
\end{document}